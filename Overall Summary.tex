\chapter{Overall Summary}
In this dissertation, we have developed a new numerical method to calculate the general five-point function, which is important for evaluating one-loop radiative corrections. Our method is developed from the magic spinor product approach in loop integrals proposed by B. F. L. Ward originally, which applied the "Chinese magic" spinor technique to simplify the loop integral so that the $E_0$ could be expressed in terms of $n$-point one-loop integrals $(n\leq4)$. And the $n$-point one-loop integrals $(n\leq4)$ can be calculated numerically by the package LoopTools. Theoretically, the magic spinor product method should provide more efficiency and numerical stability for the evaluation of the general five point function. By comparing the results obtained by our method with those directly obtained from LoopTools, we find that they agreed with each other overall. Such agreements are encouraging.

Additionally, we also developed an approach to achieve the next-to-leading order and the electroweak (EW) exact $O(\alpha_s\otimes\alpha^2L)$ corrections, interfacing MG5\textunderscore aMC@NLO with KKMC-hh by merging their LHE files. We first coded a program to read the event information from the LHE file of MG5\textunderscore aMC@NLO, and then extracted the next-to-leading QCD $O(\alpha_s)$ correction. Combining the NLO QCD corrections computed by MG5\textunderscore aMC@NLO with the basic weight for generating events in the KKMC-hh, we obtained a new basic weight including both the NLO QCD $O(\alpha_s)$ corrections and the EW $O(\alpha^2L)$ corrections. With the help of new basic weight, the new events with $O(\alpha_s\otimes\alpha^2L)$ were generated. We compared the muon transverse momentum distributions, muon pseudorapidity distributions, dimuon invariant mass distributions, dimuon rapidity distributions obtained by KKMC-hh, MG5\textunderscore aMC@NLO and KKMC-hh interfaced with MG5\textunderscore aMC@NLO , at $\sqrt{s}=13\text{ TeV}$ with the ATLAS cuts on the $Z/\gamma^\ast$ production and decay to lepton pairs, respectively. By comparing the results of the Drell-Yan process obtained by these three generators, we find that the results derived from KKMC-hh interfaced with MG5\textunderscore aMC@NLO would bring enhancements from those derived from MG5\textunderscore aMC@NLO, which is due to the EW corrections provided by KKMC-hh. We conclude that interfacing MG5\textunderscore aMC@NLO with KKMC-hh would provide a way to achieve the $O(\alpha_s\otimes\alpha^2L)$ corrections. 
 