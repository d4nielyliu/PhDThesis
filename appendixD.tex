\chapter{The Global Positioning of Spin GPS scheme}
In Chapter 3, we introduced the Kleiss and Stirling spinor method. In the KS scheme, the massless spinors and massive spinors are defined \cite{KS}. The definitions in the KS scheme will be supplemented in Ref. \cite{GPS} with the precise prescription of the spin quantization axes, the translation from spin amplitudes to density matrices, and the methodology of connecting production and decay for unstable fermions. 

The GPS rules determining the spin quantization frame for the $u(p,\pm)$ and $v(p,\pm)$ of eq. (3.99) are summarized as follows:

(\romannumeral 1) In the rest frame of the fermion, take the $z$-axis along $-\vec{k}$.

(\romannumeral 2) Place the $x$ axis in the plane define by the $z$-axis from the previous point and the vector $\vec{\eta}$, in the same half-plane as $\vec{\eta}$.

(\romannumeral 3) With the $y$-axis, complete the right-handed system of coordinates. The rest frame defined in this way we call the GPS frame of the particular fermion.

Next we will assume that polarization vectors of beams and of outgoing fermions are defined in their corresponding GPS frames.

For the definitions of inner product of the spinors are the same as those described in Chapter 3. 

For a circularly polarization vector with four-momentum $k$ and helicity $\sigma=\pm 1$ we take the following convention \cite{ChnMag}:
\begin{equation*}
[\epsilon^\mu_\sigma(\beta)]^\ast=\frac{\bar{u}_\sigma(k)\gamma^\mu u_\sigma(\beta)}{\sqrt{2}u_{-\sigma}(k)u_\sigma(\beta)}
\end{equation*}
\begin{equation}
[\epsilon^\mu_\sigma(\zeta)]^\ast=\frac{\bar{u}_\sigma(k)\gamma^\mu u_\sigma(\zeta)}{\sqrt{2}u_{-\sigma}(k)u_\sigma(\zeta)}
\end{equation} 
where  $\beta$ is an arbitrary light-like four-vector $\beta^2=0$. The second choice with $u_\sigma(\zeta)$ (constant basic spinors) often simplifies the resulting photon emission amplitudes. With the help of the Chisholm identity
\begin{equation}
\bar{u}_\sigma(k)\gamma_\mu u_\sigma(\beta)\gamma^\mu=2u_\sigma(\beta)\bar{u}_{-\sigma}(k)+2u_\sigma(k)\bar{u}_{-\sigma}(\beta),
\end{equation}
\begin{equation}
\bar{u}_\sigma(k)\gamma_\mu u_\sigma(\zeta)\gamma^\mu=2u_\sigma(\zeta)\bar{u}_{-\sigma}(k)+2u_\sigma(k)\bar{u}_{-\sigma}(\zeta),
\end{equation}
we obtain two useful formula, equivalent to eq. (D.1)
\begin{eqnarray}
\slashed\epsilon^\ast_\sigma(k,\beta)&=&\frac{\sqrt{2}}{\bar{u}_{-\sigma}(k)u_\sigma(\beta)}[u_\sigma(\beta)\bar{u}_{-\sigma}(k)+u_\sigma(k)\bar{u}_{-\sigma}(\beta)],\nonumber\\
\slashed\epsilon^\ast_\sigma(k,\zeta)&=&\frac{\sqrt{2}}{\sqrt{2\zeta k}}[u_\sigma(\zeta)\bar{u}_{-\sigma}(k)-u_\sigma(k)\bar{u}_{-\sigma}(\zeta)].
\end{eqnarray}

While calculating photon emission spin amplitudes, we will use the following important building blocks, i.e., the elements of the ``transition matrices" $U$ and $V$ defined as 
\begin{eqnarray}
\bar{u}(p_1,\lambda_1)\slashed\epsilon^\ast_\sigma(k,\beta)u(p_2,\lambda_2)=U\binom{k}{\sigma}\left[\begin{array}{c}
p_1p_2\\\lambda_1\lambda_2
\end{array}\right]=U^\sigma_{\lambda_1,\lambda_2}(k,p_1,m_1,p_2,m_2),\nonumber\\
\bar{v}(p_1,\lambda_1)\slashed\epsilon^\ast_\sigma(k,\zeta)v(p_2,\lambda_2)=V\binom{k}{\sigma}\left[\begin{array}{c}
p_1p_2\\\lambda_1\lambda_2
\end{array}\right]=V^\sigma_{\lambda_1,\lambda_2}(k,p_1,m_1,p_2,m_2).\nonumber\\
\end{eqnarray}
In the case of $u_\sigma(\zeta)$ the above transition matrices reads
\begin{equation}
U^+(k,p_1,m_1,p_2,m_2)=\sqrt{2}\left[\begin{array}{cc}
\sqrt{\frac{2\zeta p_2}{2\zeta k}}s_+(k,\hat{p}_1),&0\\
m_2\sqrt{\frac{2\zeta p_1}{2\zeta p_2}}-m_1\sqrt{\frac{2\zeta p_1}{2\zeta p_2}},&\sqrt{\frac{2\zeta p_1}{2\zeta k}}s_+(k,\hat{p}_2)
\end{array}\right],
\end{equation}
\begin{equation}
U^-_{\lambda_1,\lambda_2}(k,p_1,m_1,p_2,m_2)=[-U^+_{\lambda_2,\lambda_1}(k,p_2,m_2,p_1,m_1)]^\ast,
\end{equation}
\begin{equation}
V^\sigma_{\lambda_1,\lambda_2}(k,p_1,m_1,p_2,m_2)=-U^\sigma_{-\lambda_1,-\lambda_2}(k,p_1,-m_1,p_2,-m_2).
\end{equation}

Compared with the case of $u_\sigma(\zeta)$, the more general case $u(\beta)$ is a little bit more complicated
\begin{eqnarray}
&&U^+(k,p_1,m_1,p_2,m_2)\nonumber\\
&=&\sqrt{\frac{2}{s_-(k,\beta)}}\nonumber\\
&&\times\left[\begin{array}{cc}
s_+(\hat{p}_1,k)s_-(\beta,\hat{p}_2)+m_1m_2\sqrt{\frac{2\zeta\beta}{2\zeta p_1}\frac{2\zeta k}{2\zeta p_2}}, m_1\sqrt{\frac{2\zeta\beta}{2\zeta p_1}}s_+(k,\hat{p}_2)+m_2\sqrt{\frac{2\zeta\beta}{2\zeta p_2}s_+(\hat{p}_1,k)}\\
m_1\sqrt{\frac{2\zeta k}{2\zeta p_1}}s_-(\beta,\hat{p}_2)+m_2\sqrt{\frac{2\zeta k}{2\zeta p_2}s_-(\hat{p}_1,\beta)}, s_-(\hat{p}_1,\beta)s_+(k,\hat{p}_2)+m_1m_2\sqrt{\frac{2\zeta\beta}{2\zeta p_1}\frac{2\zeta k}{2\zeta p_2}}
\end{array}\right].\nonumber\\
\end{eqnarray} 
The numbering of elements in matrices $U$ and $V$ is
\begin{equation}
\{(\lambda_1,\lambda_2)\}=\left[\begin{array}{cc}
(++)&(+-)\\
(-+)&(--)
\end{array}\right].
\end{equation}

When computing bremsstrahlung amplitudes we will adopt the following compact notation:
\begin{eqnarray}
U\left[\begin{array}{c}
pkp\\\lambda_1\sigma\lambda_2
\end{array}\right]&\equiv&U^\sigma_{\lambda_1,\lambda_2}(k,p_1,m_1,p_2,m_2)\nonumber\\
V\left[\begin{array}{c}
pkp\\\lambda_1\sigma\lambda_2
\end{array}\right]&\equiv&V^\sigma_{\lambda_1,\lambda_2}(k,p_1,m_1,p_2,m_2).
\end{eqnarray}

When dealing with the soft real photon limit we will implement the following important diagonality property:
\begin{equation}
U\left[\begin{array}{c}
pkp\\\lambda_1\sigma\lambda_2
\end{array}\right]=V\left[\begin{array}{c}
pkp\\\lambda_1\sigma\lambda_2
\end{array}\right]=b_\sigma(k,p)\delta_{\lambda_1,\lambda_2},
\end{equation}
\begin{equation}
b_\sigma(k,p)=\sqrt{2}\frac{\bar{u}(k)\slashed p u_\sigma(\zeta)}{\bar{u}(k)u_\sigma(\zeta)}=\sqrt{2}\sqrt{\frac{2\zeta p}{2\zeta k}}s_\sigma(k,\hat{p}),
\end{equation}
which also holds in the general case of $u_\sigma(\beta)$, where
\begin{equation}
b_\sigma(k,p)=\frac{\sqrt{2}}{s_{-\sigma}(k,\beta)}\biggl( s_{-\sigma}(\beta,\hat{\beta})s_\sigma(\hat{p},k)+\frac{m^2}{2\zeta\hat{p}} \sqrt{(2\beta\zeta)(2\zeta k)}\biggr).
\end{equation}